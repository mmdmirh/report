\documentclass[12pt]{article}
\usepackage[utf8]{inputenc}
\usepackage[T1]{fontenc}
\usepackage{lmodern}
\usepackage{geometry}
\geometry{a4paper, margin=1in}
\usepackage{setspace}
\onehalfspacing

\title{AI Tool Usage Disclosure\\
\large FitCoachAR: Real-Time Adaptive Exercise Coaching}
\author{Yangyang Zhang \and Maximilian Fuchs \and Seyedmohamad Mirhoseininejad}
\date{CAS 772 -- Mobile Data Analytics \\ December 2025}

\begin{document}
\maketitle

\section*{AI Tool Usage Disclosure}

In accordance with the course AI Use Policy, this document discloses all uses of AI tools in the preparation of the project report and code.

\section{Writing Assistance}

\paragraph{Tool:} Google Gemini 2.0 Pro (via Cursor IDE)

\paragraph{Date:} December 2025

\paragraph{Purpose:} Grammar correction, clarity improvements, and academic style refinement for the Discussion section.

\paragraph{Prompt Used:}
\begin{verbatim}
Correct the grammatical errors of the following 
paragraphs and use advanced grammar:

The FormScript feedback system, while effective 
for real-time coaching, have several inherent 
limitations. First, the current implementation 
reliance on manual-defined thresholds...
[full draft text provided]
\end{verbatim}

\paragraph{Outcome:} The AI suggested corrections for subject-verb agreement, word choice, and sentence structure. All suggestions were reviewed and selectively incorporated. No substantive content was generated by the AI.

\section{Code Generation}

\subsection{RepCount Dataset Parser}

\paragraph{Tool:} Google Gemini 2.0 Pro

\paragraph{Date:} December 2025

\paragraph{Purpose:} Generate a Python class to parse the RepCount (LLSP) dataset CSV annotations for FormCode validation experiments.

\paragraph{Prompt Used:}
\begin{verbatim}
Write a Python class to parse the RepCount (LLSP) 
dataset CSV annotations. The solution must use 
dataclasses to represent a single repetition 
(RepAnnotation with start/end frames and index) 
and a complete video's annotations (VideoAnnotation 
including exercise type, video name, total rep count, 
and a list of RepAnnotation objects).

The main parser class, RepCountDatasetParser, should:
1. Initialize with the dataset's base path
2. Have a parse_split(split: str) method
3. Implement _extract_reps(row) for frame boundaries
4. Implement _find_video_path(video_name, split)
5. Include filter_by_exercise() with name variations
\end{verbatim}

\paragraph{Outcome:} The AI generated a complete parser implementation using Python dataclasses. The code was reviewed, tested on the actual dataset, and integrated into \texttt{dataset\_parser.py}. Minor modifications were made for path handling and error cases.

\subsection{TikZ Block Diagram}

\paragraph{Tool:} Google Gemini 2.0 Pro

\paragraph{Date:} December 2025

\paragraph{Purpose:} Generate LaTeX TikZ code for the FormScript pipeline visualization.

\paragraph{Prompt Used:}
\begin{verbatim}
Create a TikZ block diagram with:
- Horizontal Flow: Video Frames -> Pose Estimation 
  -> FormCodes -> Super FormCodes
- Vertical Drop: Super FormCodes -> Form Analyzer
- Horizontal Flow: Form Analyzer -> LLM -> Feedback
\end{verbatim}

\paragraph{Outcome:} The AI generated TikZ code which was adapted and integrated into the methodology section figures. Node styling and positioning were manually adjusted.

\section{Verification Statement}

All AI-generated code was:
\begin{enumerate}
    \item Reviewed line-by-line for correctness
    \item Tested on actual project data
    \item Modified as necessary for integration
    \item Verified to produce expected outputs
\end{enumerate}

The author remains fully responsible for all content, code functionality, and any errors in the report.

\end{document}
