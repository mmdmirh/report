\documentclass[12pt]{article}
\usepackage[utf8]{inputenc}
\usepackage[T1]{fontenc}
\usepackage{lmodern}
\usepackage{geometry}
\geometry{a4paper, margin=1in}
\usepackage{setspace}
\onehalfspacing

\title{AI Tool Usage Disclosure\\
\large FitCoachAR: Real-Time Adaptive Exercise Coaching}
\author{Yangyang Zhang \and Maximilian Fuchs \and Seyedmohamad Mirhoseininejad}
\date{CAS 772 -- Mobile Data Analytics \\ December 2025}

\begin{document}
\maketitle

\section*{AI Tool Usage Disclosure}

In accordance with the course AI Use Policy, this document discloses all uses of AI tools in the preparation of the project report and code.

\section{Writing Assistance}

\paragraph{Tool:} Google Gemini 2.0 Pro (via Cursor IDE)


\paragraph{Purpose:} Grammar correction, clarity improvements, and academic style refinement for the Discussion section.

\paragraph{Prompt Used:}
\begin{verbatim}
Correct the grammatical errors of the following 
paragraphs and use advanced grammar:

The FormScript feedback system, while effective 
for real-time coaching, have several inherent 
limitations. First, the current implementation 
reliance on manual-defined thresholds...
[full draft text provided]
\end{verbatim}

\paragraph{Outcome:} The AI suggested corrections for subject-verb agreement, word choice, and sentence structure. All suggestions were reviewed and selectively incorporated. No substantive content was generated by the AI.

\subsection{Post-Processing Results Description}

\paragraph{Tool:} Google Gemini 3 Pro (thinking)


\paragraph{Purpose:} Polish a results paragraph for clarity and academic tone.

\paragraph{Prompt Used:}
\begin{verbatim}
Please help polish the following paragraph for 
clarity and academic tone, without changing its 
technical meaning:

"Table 3 shows that MoveNet is fast but not stable 
across views. After adding our post-processing, the 
variance becomes much smaller, especially for elbow 
and knee joints."
\end{verbatim}

\paragraph{Outcome:} The AI refined the paragraph to:
\begin{quote}
``Table 3 indicates that while MoveNet achieves low inference latency, its predictions exhibit noticeable instability across different viewpoints. After applying the proposed post-processing pipeline, the variance is substantially reduced, particularly for view-sensitive joints such as the elbows and knees.''
\end{quote}
This polished version was incorporated into the results section.

\section{Code Generation}

\subsection{RepCount Dataset Parser}

\paragraph{Tool:} Google Gemini 2.0 Pro


\paragraph{Purpose:} Generate a Python class to parse the RepCount (LLSP) dataset CSV annotations for FormCode validation experiments.

\paragraph{Prompt Used:}
\begin{verbatim}
Write a Python class to parse the RepCount (LLSP) 
dataset CSV annotations. The solution must use 
dataclasses to represent a single repetition 
(RepAnnotation with start/end frames and index) 
and a complete video's annotations (VideoAnnotation 
including exercise type, video name, total rep count, 
and a list of RepAnnotation objects).

The main parser class, RepCountDatasetParser, should:
1. Initialize with the dataset's base path
2. Have a parse_split(split: str) method
3. Implement _extract_reps(row) for frame boundaries
4. Implement _find_video_path(video_name, split)
5. Include filter_by_exercise() with name variations
\end{verbatim}

\paragraph{Outcome:} The AI generated a complete parser implementation using Python dataclasses. The code was reviewed, tested on the actual dataset, and integrated into \texttt{dataset\_parser.py}. Minor modifications were made for path handling and error cases.

\subsection{TikZ Block Diagram}

\paragraph{Tool:} Google Gemini 2.0 Pro


\paragraph{Purpose:} Generate LaTeX TikZ code for the FormScript pipeline visualization.

\paragraph{Prompt Used:}
\begin{verbatim}
Create a TikZ block diagram with:
- Horizontal Flow: Video Frames -> Pose Estimation 
  -> FormCodes -> Super FormCodes
- Vertical Drop: Super FormCodes -> Form Analyzer
- Horizontal Flow: Form Analyzer -> LLM -> Feedback
\end{verbatim}

\paragraph{Outcome:} The AI generated TikZ code which was adapted and integrated into the methodology section figures. Node styling and positioning were manually adjusted.

\subsection{EMA Smoothing Function}

\paragraph{Tool:} Claude Sonnet 4.5 (Anthropic)


\paragraph{Purpose:} Generate a simple exponential moving average function for keypoint smoothing.

\paragraph{Prompt Used:}
\begin{verbatim}
I have 2D keypoints from MoveNet in normalized image 
coordinates. I want to apply exponential moving average 
(EMA) smoothing to reduce jitter, but I want the 
smoothing factor to be adjustable. Please generate a 
simple Python function for this.
\end{verbatim}

\paragraph{Outcome:} The AI generated the following function:
\begin{verbatim}
def ema_smooth(curr, prev, alpha=0.8):
    """
    Exponential Moving Average smoothing.
    curr: current keypoints (N x 2)
    prev: previous smoothed keypoints (N x 2)
    alpha: smoothing factor
    """
    if prev is None:
        return curr
    return alpha * prev + (1 - alpha) * curr
\end{verbatim}
This function was integrated into the post-processing pipeline.

\subsection{Form Analyzer Debugging}

\paragraph{Tool:} Google Gemini 2.0 Pro


\paragraph{Purpose:} Debug an error in the \texttt{form\_analyzer.py} module.

\paragraph{Context:} The FormAnalyzer class was implemented, but the \texttt{categorize\_form\_code()} method was returning incorrect categories for edge cases.

\paragraph{Prompt Used:}
\begin{verbatim}
I have the following error in my form_analyzer.py:

TypeError: '<' not supported between instances 
of 'NoneType' and 'float'

The error occurs in categorize_form_code() when 
checking threshold conditions. Here is the method:

def categorize_form_code(self, exercise, name, value):
    config = FORM_CODES_CONFIG[exercise]
    for cat in config[name]["categories"]:
        if value < cat.get("v_max"):
            return cat["name"]
    return None

How can I fix this to handle missing v_max values?
\end{verbatim}

\paragraph{Outcome:} The AI suggested adding a conditional check for \texttt{None} values:
\begin{verbatim}
v_max = cat.get("v_max")
if v_max is not None and value < v_max:
    return cat["name"]
\end{verbatim}
This fix was integrated into the existing code. The core logic and architecture of \texttt{form\_analyzer.py} (419 lines) was developed independently.

\section{Verification Statement}

All AI-generated code was:
\begin{enumerate}
    \item Reviewed line-by-line for correctness
    \item Tested on actual project data
    \item Modified as necessary for integration
    \item Verified to produce expected outputs
\end{enumerate}

The author remains fully responsible for all content, code functionality, and any errors in the report.

\end{document}
