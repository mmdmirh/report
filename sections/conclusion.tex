\section{Conclusion}

This project presented \textbf{FitCoachAR}, a real-time, adaptive exercise coaching system that bridges the gap between academic pose-analysis pipelines and practical consumer fitness applications.

Through a comprehensive empirical evaluation, we showed that among a wide range of pose estimation backends,
\textbf{MoveNet\_2D} offers the most favorable accuracy--latency trade-off for real-time use.
More importantly, our four-step post-processing pipeline—body-centric canonicalization,
yaw compensation, confidence-/motion-adaptive temporal smoothing, and anatomical consistency projection—
substantially improves both joint-angle accuracy and cross-view stability without increasing inference latency.

Beyond low-level pose accuracy, FitCoachAR introduces \textbf{FormScript}, a semantic intermediate representation that transforms continuous pose measurements into interpretable FormCodes.
This abstraction enables exercise-specific reasoning, personalized calibration, and natural-language feedback,
supporting both real-time corrective cues and post-workout summaries.
By decoupling biomechanical definitions from pose estimation and feedback generation,
FormScript provides a flexible and extensible framework for future exercise analysis systems.

Overall, FitCoachAR illustrates a practical design philosophy for interactive human-motion applications:
prioritizing robustness, interpretability, and responsiveness over raw model complexity.
While several limitations remain—including the lack of ablation analysis and the reliance on aggregated metrics—
the results suggest that integrated, lightweight post-processing can significantly narrow the gap
between small 2D models and heavier 3D approaches in real-world fitness coaching scenarios.

Future work may explore finer-grained phase-specific evaluation, user studies on coaching effectiveness,
and the extension of FormCodes to rehabilitation and clinical movement assessment.
Nonetheless, this project demonstrates that real-time, personalized, and explainable exercise coaching
is achievable on consumer hardware using principled pose processing and semantic reasoning.
