\subsection{Feature Computation and Calibration}
From each repetition we compute angular features:
\begin{itemize}
    \item Active joints: elbows, knees, shoulders (max, min, correlation).
    \item Passive joints: spine and pelvis (mean, standard deviation).
\end{itemize}

During calibration, each key joint or motion phase $i$ is analyzed over three ``best-form'' repetitions:
The mean joint angle is
\begin{equation}
    \bar{U}_i = \frac{1}{3} \sum_{r=1}^{3} \theta_i^{(r)}
\end{equation}

Its deviation percentage from a canonical target $S_i$ is
\begin{equation}
    \eta_i = \frac{\\bar{U}_i - S_i}{S_i}
\end{equation}

The pair $(\bar{U}_i, \eta_i)$ forms the personalized baseline and offset for that user. Specifically, for Squats, the Knee Joint angle is calibrated to define the personalized depth threshold ($\theta_{depth}$). For Bicep Curls, the Elbow Joint angle is calibrated to define the full contraction threshold ($\theta_{peak}$). Other metrics (e.g., stability limits) rely on non-personalized biomechanical defaults to ensure safety.
During runtime, for each joint or phase, the system:
\begin{enumerate}
    \item Measures the current angle $\theta_i$
    \item Computes percentage deviation $e_i = \frac{|\theta_i - \bar{U}_i|}{|\bar{U}_i|}$
\end{enumerate}

A manually set critic parameter $\delta$ determines grading bands:
\begin{itemize}
    \item Good: $e_i < \delta$
    \item Relatively good: $\delta \le e_i < 1.2 \delta$
    \item Bad: $e_i \ge 1.2 \delta$
\end{itemize}

Repetition-level and session-level scores aggregate these joint/phase grades to summarize overall performance.
