\subsection{Feature Computation and Calibration}

Calibration is performed by analyzing a short recording of continuous exercise motion and
operates directly on per-repetition extrema rather than predefined canonical poses.
For each exercise, a primary kinematic signal is selected (e.g., elbow flexion for bicep curls,
normalized hip height for squats), along with a set of auxiliary form-related metrics.

From the calibration sequence, repetitions are first segmented using extrema of the primary
signal. For each detected repetition $r$, the system extracts:
\begin{itemize}
    \item \textbf{Primary features}: bottom and top extrema values of the primary signal,
    repetition amplitude, and duration.
    \item \textbf{Form metrics}: joint angles, inter-joint distances, symmetry indices, and
    posture-related measures computed either per frame or aggregated per repetition.
\end{itemize}

Across all valid repetitions, the system computes aggregate statistics for each metric.
For the primary signal, the bottom and top extrema are summarized as:
\begin{equation}
    \theta_{\text{low}} = \operatorname{median}\{b_r\}, \qquad
    \theta_{\text{high}} = \operatorname{median}\{p_r\},
\end{equation}
where $b_r$ and $p_r$ denote the bottom and top values of repetition $r$.
These values define the personalized repetition detection thresholds.

For each form metric $m$, the calibration produces normative statistics:
\begin{equation}
    m_{\text{norm}} = \operatorname{median}\{m_r\}, \quad
    \mu_m = \mathbb{E}[m_r], \quad
    \sigma_m = \sqrt{\mathbb{E}[(m_r-\mu_m)^2]},
\end{equation}
along with observed minimum and maximum values.
These statistics characterize the user’s typical movement pattern and variability during
the calibration session.

Unlike approaches that rely on fixed biomechanical targets, all thresholds and normative
references are inferred directly from the user’s own repetitions. The resulting calibration
parameters consist of:
\begin{itemize}
    \item Primary repetition thresholds $(\theta_{\text{low}}, \theta_{\text{high}})$ for counting.
    \item Normative form references $m_{\text{norm}}$ and dispersion measures $(\mu_m, \sigma_m)$
    for form evaluation.
\end{itemize}