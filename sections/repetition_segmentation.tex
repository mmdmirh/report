\subsection{Online Repetition Segmentation and Form Evaluation}

During live operation, repetition detection and form evaluation are decoupled but driven by
the same calibration parameters.

\paragraph{Repetition Detection.}
An online state machine operates on the primary signal and uses the calibrated thresholds
$\theta_{\text{low}}$ and $\theta_{\text{high}}$ to detect transitions between movement phases
(e.g., bottom, ascent, top, descent). A full traversal from bottom to top and back to bottom
is labeled as one repetition. This stage is intentionally form-agnostic and focuses solely on
robust cycle detection.

\paragraph{Form Check and Rep Acceptance.}
For each detected repetition, form metrics are evaluated against their calibrated normative
references. Given a live metric value $m$ and its normative reference $m_{\text{norm}}$, the
relative deviation is computed as:
\begin{equation}
    e = \frac{|m - m_{\text{norm}}|}{|m_{\text{norm}}|}.
\end{equation}

A user-controlled critic parameter $\delta$ scales the acceptable tolerance for all form
metrics. For a detected repetition to be considered \emph{valid}, every evaluated form
metric must satisfy the deviation constraint. Specifically, if there exists any metric
for which
\begin{equation}
    e > \delta,
\end{equation}
the repetition is marked as invalid and is not counted, even if the kinematic cycle was
successfully detected.

Only repetitions for which all form metrics satisfy $e \le \delta$ increment the repetition
counter.