\section{Qualitative Results}
\label{sec:results}

We conducted a live demonstration of the FitCoachAR system to validate its core components: real-time visual feedback, quantitative form analysis, and generative AI coaching. The following results illustrate the system's output during a standard workout session (Bicep Curls and Squats).

\subsection{Real-Time Visual Feedback}
The primary contribution of FitCoachAR is the low-latency Augmented Reality overlay. Figure \ref{fig:ar_arrows} demonstrates the system's ability to detect form errors in real-time. In this example (Bicep Curl), the system detected "Elbow Drift" and immediately rendered a directional arrow guiding the user to tuck their elbow in. This visual cue is generated within 100ms, providing instant correction without interrupting the exercise flow.

\begin{figure}[h]
    \centering
    \includegraphics[width=0.8\textwidth]{figures/arrow.png}
    \caption{Real-time AR feedback mechanism. The system detects an elbow stability error and superimposes a yellow arrow on the user's arm, directing them to correct their form.}
    \label{fig:ar_arrows}
\end{figure}

\subsection{Quantitative Form Analysis}
After completing a set, the system compiles a detailed "Form Analysis" report. Figure \ref{fig:form_analysis} shows the post-set summary screen. The system aggregates data from all repetitions to calculate an overall "Form Score" (e.g., 90\%). It also provides a breakdown of specific issues, such as "Body Swing" or "Incomplete Range," along with a rep-by-rep analysis. This allows users to identify consistent patterns in their technique.

\begin{figure}[h]
    \centering
    \includegraphics[width=0.8\textwidth]{figures/form-analysis.png}
    \caption{Post-workout Form Analysis summary. The interface displays an overall score, a breakdown of detected errors (e.g., Body Swing), and a detailed log of each repetition.}
    \label{fig:form_analysis}
\end{figure}

\subsection{Generative AI Coaching}
To provide deeper insights, FitCoachAR integrates a Large Language Model (LLM) as a virtual coach. Figure \ref{fig:ai_coach} shows the chat interface where the user can ask questions about their performance. The LLM receives the structured workout data (FormCodes) and provides natural language advice. In this instance, the coach offers specific tips on how to fix the "Body Swing" issue detected during the set.

\begin{figure}[h]
    \centering
    \includegraphics[width=0.8\textwidth]{figures/ai-response.png}
    \caption{Generative AI Coach interface. The user interacts with the LLM to receive personalized, actionable advice based on the workout data.}
    \label{fig:ai_coach}
\end{figure}

\subsection{Data Processing pipeline}
Underlying these visual interfaces is a robust data processing pipeline. Figure \ref{fig:script_response} visualizes the raw JSON outputs generated by the FormScript engine. These structured responses confirm that the system correctly translates raw pose landmarks into semantic "Super FormCodes" (e.g., \texttt{GOOD\_REP}, \texttt{ELBOW\_DRIFT}), which then drive both the AR arrows and the AI coaching.

\begin{figure}[h]
    \centering
    \includegraphics[width=0.8\textwidth]{figures/script-responses.png}
    \caption{Backend data processing logs. The JSON output confirms the detection of specific FormCodes, verifying the accuracy of the rule-based evaluation engine.}
    \label{fig:script_response}
\end{figure}
