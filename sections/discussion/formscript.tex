\subsection{FormScript and LLM Coaching Limitations}

The FormScript feedback system, while effective for real-time coaching, has several inherent limitations.
First, the current implementation relies on manually-defined thresholds for FormCode categorization, which may not generalize across all users or exercises without personalized calibration.
Although we demonstrated data-driven threshold optimization on the RepCount dataset, this process requires access to labeled ground-truth data that may not be available for new exercises.

Second, the Super FormCode definitions are currently hardcoded for specific exercises (squats and bicep curls).
Extending the system to new exercises requires domain expertise to define appropriate biomechanical rules and feedback mappings, limiting rapid scalability.
Future work should explore learned representations or evolutionary approaches to automate this process.

Third, the LLM-powered post-workout summaries depend on the quality and completeness of the per-rep feedback data.
While we send all FormScript sentences to the LLM for context, the model's response quality is ultimately bounded by the information captured during the session.
The LLM also operates as a black box, making it difficult to guarantee consistency or avoid occasionally generic advice.

Finally, our FormCode validation was limited to squat exercises from a single dataset.
Generalization to other exercises, user populations, or recording conditions remains to be validated.
Despite these limitations, FormScript demonstrates that rule-based semantic abstractions combined with LLM reasoning provide a practical path toward explainable and personalized fitness coaching.
