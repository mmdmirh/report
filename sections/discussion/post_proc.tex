\subsection{Biases and Limitations in Model Comparison and Post-Processing Evaluation}

While the proposed post-processing approach demonstrates strong robustness--latency trade-offs
for real-time fitness coaching, several limitations should be acknowledged.
First, our evaluation projects both ground-truth 3D joints and predictions from
3D-based models into a normalized 2D torso-centric plane to enable fair comparison
with 2D pose estimators.
Although this design removes viewpoint bias, it may also suppress the inherent
depth advantages of true 3D models, potentially underestimating their full capability.
Second, discrepancies in joint definitions across datasets and models introduce
additional uncertainty: ground-truth annotations are derived from sensor-based
motion-capture systems, whose joint semantics do not perfectly align with those of
learning-based 2D estimators such as MoveNet, MediaPipe, or ROMP.

Third, the four proposed post-processing techniques were evaluated as a combined
pipeline, without a dedicated ablation study to isolate the individual contribution
of each component.
While this choice reflects our system-oriented focus on end-to-end performance,
it limits fine-grained attribution of improvements.
Finally, the high dimensionality of pose data—spanning multiple joints, frames,
viewpoints, and exercises—was summarized using global metrics.
Such aggregation may obscure phase-specific or viewpoint-specific behaviors,
particularly for exercises where joint dynamics vary substantially across motion phases.
