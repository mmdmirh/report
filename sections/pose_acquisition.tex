\subsection{Pose Estimation Model Choice and Post-Processing}
\label{sec:movenet-postproc}
We benchmarked MediaPipe~2D/3D, MoveNet~2D, HRNet~2D (+ lifters), and ROMP~3D, and chose \textbf{MoveNet~2D} for its accuracy--latency balance. To further stabilize its output, we introduce four lightweight techniques applied per frame. We present each step with intuition and a compact formulation, allowing more breathing room than a terse bullet list.

\paragraph{Body-centric Canonicalization.}
Let $p_j\in\mathbb{R}^2$ be raw keypoints; torso center $c=\frac{1}{4}(p_{LS}+p_{RS}+p_{LH}+p_{RH})$. We translate $\tilde p_j=p_j-c$, rotate so pelvis axis is horizontal: $\hat p_j=R(\theta)\tilde p_j$, where $\theta=\arctan2(p_{RH}-p_{LH})$. A yaw proxy $s=\frac{\|p_{RS}-p_{LS}\|}{\|p_{RH}-p_{LH}\|}$ scales $x$ (clipped to $[0.7,1.3]$): $\bar p_j=(\hat p_j^{x}/s,\hat p_j^{y})$. Finally we normalize by torso scale $t=\frac{\|p_{RS}-p_{LS}\|+\|p_{RH}-p_{LH}\|}{2}$ and map back with $R(-\theta)$. \textit{Intuition:} align to a torso-centric frame, compensate side-view foreshortening, and reduce distance effects.

\paragraph{Confidence-/Motion-Adaptive Kalman Smoothing.}
Each joint keeps a 2D Kalman state $(x,P)$; process/measurement noise adapt to confidence $c$ and motion $m$:
\[
Q = q_0 (1+2m)I,\quad R = r_0 (1+\max(0,1-c))I.
\]
Predict: $P\!\leftarrow\!P+Q$; Update: $K=P(P+R)^{-1}$, $x\!\leftarrow\!x+K(z-x)$, $P\!\leftarrow\!(I-K)P$. \textit{Intuition:} low confidence $\Rightarrow$ stronger smoothing; fast motion $\Rightarrow$ weaker smoothing to avoid lag.

\paragraph{Anatomical Consistency Projection.}
For left/right limb pairs $(a,b),(a',b')$ (upper/lower arms, thighs, calves) we target a symmetric length $L$ maintained by an EMA $L\leftarrow \alpha L + (1-\alpha)\tfrac{\|p_b-p_a\|+\|p_{b'}-p_{a'}\|}{2}$. We project $p_b,p_{b'}$ to match $L$, reducing ``rubber-arm'' artifacts while staying close to observations. \textit{Intuition:} enforce mild bone-length consistency to stabilize limbs.

\paragraph{Integrated Yaw Compensation.}
The yaw scaling above doubles as a cheap side-view correction: broader shoulders vs. hips imply less foreshortening; the anisotropic scaling mitigates view-induced distortion without extra passes. \textit{Intuition:} fix side-view distortion without extra latency.

These steps add negligible overhead and, as shown in Sec.~\ref{sec:results}, significantly tighten joint-angle/distance variance while keeping per-frame latency essentially unchanged.
