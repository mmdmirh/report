\subsection{FormScript: Interpretable Feedback via FormCodes}
\label{sec:formscript}

To provide meaningful, explanatory feedback, we implemented ``FormScript'', a rule-based system inspired by the ``PoseScript'' methodology \cite{delmas2022posescript}. The core problem is that simple rep counters don't tell users \textit{how} to improve---users need to know \textit{why} a rep was good or bad. FormScript solves this by discretizing continuous kinematic measurements into human-readable categories (FormCodes) and then applying logical production rules (Super FormCodes) to generate actionable feedback.

\subsubsection{Elementary FormCodes}

Following PoseScript's taxonomy, we define five types of elementary FormCodes. Each FormCode transforms a continuous measurement $v$ (angle in degrees or distance in meters) into a discrete category based on thresholds.

\paragraph{Angle FormCodes}
Measure joint flexion at key moments during exercise.

\begin{table}[h]
\centering
\small
\begin{tabular}{|l|l|l|}
\hline
\textbf{Categorization} & \textbf{Condition} & \textbf{Applied To} \\
\hline
completely bent & $v \pm 5 \leq 45$ & L/R-knee, L/R-elbow \\
almost completely bent & $45 < v \pm 5 \leq 75$ & L/R-knee, L/R-elbow \\
bent at right angle & $75 < v \pm 5 \leq 105$ & L/R-knee, L/R-elbow \\
partially bent & $105 < v \pm 5 \leq 135$ & L/R-knee, L/R-elbow \\
slightly bent & $135 < v \pm 5 \leq 160$ & L/R-knee, L/R-elbow \\
straight & $v \pm 5 > 160$ & L/R-knee, L/R-elbow \\
\hline
\end{tabular}
\caption{Angle FormCode categorizations with noise tolerance $\pm 5^\circ$.}
\label{tab:angle-formcodes}
\end{table}

\paragraph{Distance FormCodes}
Measure relative spacing between body parts.

\begin{table}[h]
\centering
\small
\begin{tabular}{|l|l|l|}
\hline
\textbf{Categorization} & \textbf{Condition (m)} & \textbf{Applied To} \\
\hline
close & $v \pm 0.05 \leq 0.20$ & L/R-elbow vs. torso \\
shoulder width & $0.20 < v \pm 0.05 \leq 0.40$ & L/R-foot, L/R-hand \\
spread & $0.40 < v \pm 0.05 \leq 0.80$ & L/R-knee, L/R-foot \\
wide & $v \pm 0.05 > 0.80$ & L/R-foot stance \\
\hline
\end{tabular}
\caption{Distance FormCode categorizations.}
\label{tab:distance-formcodes}
\end{table}

\paragraph{Relative Position FormCodes}
Measure spatial relationships between joints along X, Y, and Z axes.

\begin{table}[h]
\centering
\small
\begin{tabular}{|l|l|l|}
\hline
\textbf{Axis} & \textbf{Categorization} & \textbf{Condition} \\
\hline
X (lateral) & at the right of / x-ignored / at the left of & $v \pm 0.05 \lessgtr \pm 0.15$ \\
Y (vertical) & below / y-ignored / above & $v \pm 0.05 \lessgtr \pm 0.15$ \\
Z (depth) & behind / z-ignored / in front of & $v \pm 0.05 \lessgtr \pm 0.15$ \\
\hline
\end{tabular}
\caption{Relative position FormCode categorizations along each axis.}
\label{tab:relpos-formcodes}
\end{table}

\paragraph{Pitch \& Roll FormCodes}
Measure body segment orientation relative to vertical.

\begin{table}[h]
\centering
\small
\begin{tabular}{|l|l|l|}
\hline
\textbf{Categorization} & \textbf{Condition} & \textbf{Applied To} \\
\hline
vertical (upright) & $v \pm 5 \leq 10$ & torso, pelvis \\
ignored (leaning) & $10 < v \pm 5 \leq 80$ & torso, pelvis \\
horizontal (bent over) & $v \pm 5 > 80$ & torso, pelvis \\
\hline
\end{tabular}
\caption{Pitch \& roll FormCode categorizations.}
\label{tab:pitchroll-formcodes}
\end{table}

\paragraph{Ground-Contact FormCodes}
Detect whether body parts are in contact with the ground.

\begin{table}[h]
\centering
\small
\begin{tabular}{|l|l|l|}
\hline
\textbf{Categorization} & \textbf{Condition (m)} & \textbf{Applied To} \\
\hline
on the ground & $v \pm 0.05 \leq 0.10$ & L/R-knee, L/R-foot \\
ground-ignored & $v \pm 0.05 > 0.10$ & L/R-knee, L/R-foot \\
\hline
\end{tabular}
\caption{Ground-contact FormCode categorizations.}
\label{tab:ground-formcodes}
\end{table}

\subsubsection{Super FormCodes: Production Rules}

Super FormCodes aggregate multiple elementary FormCodes using logical production rules. Each Super FormCode defines a high-level body configuration by combining conditions with AND/OR operators.

\begin{table}[h]
\centering
\small
\begin{tabular}{|l|l|p{6cm}|}
\hline
\textbf{Subject} & \textbf{Configuration} & \textbf{Production Rule} \\
\hline
torso & upright & pitch \& roll (pelvis, L-shoulder) = vertical AND pitch \& roll (pelvis, R-shoulder) = vertical \\
\hline
body & bent forward & relativePos Y (L-ankle, neck) = below AND relativePos Z (neck, pelvis) = front \\
\hline
knees & deep squat & angle (L-knee) = completely bent AND angle (R-knee) = completely bent \\
\hline
knees & parallel squat & angle (L-knee) = bent at right angle AND angle (R-knee) = bent at right angle \\
\hline
knees & stable & distance (L-knee, L-foot) = close AND distance (R-knee, R-foot) = close \\
\hline
knees & collapsed & relativePos X (L-knee, L-foot) = at right of OR relativePos X (R-knee, R-foot) = at left of \\
\hline
elbows & anchored & distance (L-elbow, torso) = close AND distance (R-elbow, torso) = close \\
\hline
elbows & drifting & distance (L-elbow, torso) = spread OR distance (R-elbow, torso) = spread \\
\hline
feet & shoulder width & distance (L-foot, R-foot) = shoulder width AND pitch \& roll (L-foot, R-foot) = horizontal \\
\hline
\end{tabular}
\caption{Super FormCode production rules for exercise analysis.}
\label{tab:super-formcodes}
\end{table}

\subsubsection{Exercise-Specific FormCode Application}

\paragraph{Squat Analysis}
Key FormCodes monitored: knee angle (depth), torso pitch (lean), knee-to-foot distance (stability), hip levelness (asymmetry).

\paragraph{Bicep Curl Analysis}
Key FormCodes monitored: elbow angle (contraction), elbow-to-torso distance (anchoring), shoulder/hip pitch (swing detection), wrist angle (neutral grip).

\subsubsection{Instant Coaching Output}

The FormScript pipeline produces real-time coaching by mapping Super FormCodes to actionable verbal cues:
\begin{itemize}
    \item knees = collapsed $\rightarrow$ ``Push your knees out''
    \item torso = bent forward $\rightarrow$ ``Chest up''
    \item knees = parallel squat $\rightarrow$ ``Go deeper''
    \item elbows = drifting $\rightarrow$ ``Pin your elbow to your side''
\end{itemize}

This hierarchical approach ensures interpretable feedback: first observing specific joint measurements, then synthesizing them into coherent, actionable coaching.
