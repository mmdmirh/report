\appendix

% Configure table numbering for appendix (A1, A2, A3, ...)
\renewcommand{\thetable}{A\arabic{table}}
\setcounter{table}{0}

\section{FormCode Reference Tables}
\label{appendix:formcodes}

This appendix provides a complete reference of all FormCode categorizations used in the FormScript system. Each table defines the discrete categories, threshold conditions, and applicable body parts for a specific FormCode type.

\subsection{Static FormCodes}

Static FormCodes evaluate body geometry on individual frames.

\subsubsection{Angle FormCodes}
Measure joint flexion at key moments during exercise.

\begin{table}[h]
\caption{Angle FormCode categorizations with noise tolerance $\pm 5^\circ$.}
\label{tab:angle-formcodes}
\centering
\small
\begin{tabular}{|l|l|l|}
\hline
\textbf{Categorization} & \textbf{Condition} & \textbf{Applied To} \\
\hline
\multicolumn{3}{|l|}{\textit{Generic Angle Ranges:}} \\
completely bent & $v \pm 5 \leq 45$ & L/R-knee, L/R-elbow \\
almost completely bent & $45 < v \pm 5 \leq 75$ & L/R-knee, L/R-elbow \\
bent at right angle & $75 < v \pm 5 \leq 105$ & L/R-knee, L/R-elbow \\
partially bent & $105 < v \pm 5 \leq 135$ & L/R-knee, L/R-elbow \\
slightly bent & $135 < v \pm 5 \leq 160$ & L/R-knee, L/R-elbow \\
straight & $v \pm 5 > 160$ & L/R-knee, L/R-elbow \\
\hline
\multicolumn{3}{|l|}{\textit{Squat Depth (Calibration-Adjusted):}} \\
deep & $v \leq 90^\circ$ & squat\_depth \\
parallel & $90^\circ < v \leq 110^\circ$ & squat\_depth \\
shallow & $v > 110^\circ$ & squat\_depth \\
\hline
\multicolumn{3}{|l|}{\textit{Bicep Curl Peak Flexion:}} \\
full\_contraction & $v \leq 60^\circ$ & peak\_flexion \\
partial\_contraction & $60^\circ < v \leq 90^\circ$ & peak\_flexion \\
incomplete\_rep & $v > 90^\circ$ & peak\_flexion \\
\hline
\end{tabular}
\end{table}

\subsubsection{Distance FormCodes}
Measure relative spacing between body parts.

\begin{table}[h]
\caption{Distance FormCode categorizations.}
\label{tab:distance-formcodes}
\centering
\small
\begin{tabular}{|l|l|l|l|}
\hline
\textbf{Categorization} & \textbf{Condition} & \textbf{Unit} & \textbf{Applied To} \\
\hline
narrow & $v < 0.9$ & ratio & stance width (feet) \\
shoulder\_width & $0.9 \leq v \leq 1.2$ & ratio & stance width (feet) \\
wide & $v > 1.2$ & ratio & stance width (feet) \\
anchored & $v \leq 0.08$ & normalized & elbow drift \\
slight\_drift & $0.08 < v \leq 0.15$ & normalized & elbow drift \\
major\_drift & $v > 0.15$ & normalized & elbow drift \\
\hline
\end{tabular}
\end{table}

\subsubsection{Relative Position FormCodes}
Measure spatial relationships between joints along X, Y, and Z axes.

\begin{table}[h]
\caption{Relative position FormCode categorizations.}
\label{tab:relpos-formcodes}
\centering
\small
\begin{tabular}{|l|l|l|}
\hline
\textbf{Measurement} & \textbf{Categorization} & \textbf{Condition} \\
\hline
Knee forward travel (Z-axis) & behind\_toes & $v \leq 0$ m \\
Knee forward travel (Z-axis) & over\_toes & $v > 0$ m \\
\hline
\end{tabular}
\end{table}

\subsubsection{Pitch \& Roll FormCodes}
Measure body segment orientation relative to vertical.

\begin{table}[h]
\caption{Pitch \& roll FormCode categorizations.}
\label{tab:pitchroll-formcodes}
\centering
\small
\begin{tabular}{|l|l|l|}
\hline
\textbf{Categorization} & \textbf{Condition} & \textbf{Applied To} \\
\hline
upright & $v \leq 45^\circ$ & torso\_angle \\
leaning & $45^\circ < v \leq 65^\circ$ & torso\_angle \\
bent\_over & $v > 65^\circ$ & torso\_angle \\
\hline
\end{tabular}
\end{table}

\subsubsection{Ground-Contact FormCodes}
Detect whether body parts are in contact with the ground.

\begin{table}[h]
\caption{Ground-contact FormCode categorizations.}
\label{tab:ground-formcodes}
\centering
\small
\begin{tabular}{|l|l|l|}
\hline
\textbf{Categorization} & \textbf{Condition (m)} & \textbf{Applied To} \\
\hline
on the ground & $v \pm 0.05 \leq 0.10$ & L/R-knee, L/R-foot \\
ground-ignored & $v \pm 0.05 > 0.10$ & L/R-knee, L/R-foot \\
\hline
\end{tabular}
\end{table}

\subsection{Dynamic FormCodes}

Dynamic FormCodes evaluate motion quality over temporal windows.

\subsubsection{Velocity FormCodes}
Measure the speed of movement phases (descent, ascent, lift, lower).

\begin{table}[h]
\caption{Velocity FormCode categorizations.}
\label{tab:velocity-formcodes}
\centering
\small
\begin{tabular}{|l|l|l|}
\hline
\textbf{Categorization} & \textbf{Condition} & \textbf{Applied To} \\
\hline
controlled & $v \leq 0.5$ m/s & descent\_speed, ascent\_speed \\
fast & $0.5 < v \leq 1.0$ m/s & descent\_speed, ascent\_speed \\
dive/explosive & $v > 1.0$ m/s & descent\_speed, ascent\_speed \\
\hline
\end{tabular}
\end{table}

\subsubsection{Jerk FormCodes}
Measure smoothness via rate of acceleration change.

\begin{table}[h]
\caption{Jerk FormCode categorizations.}
\label{tab:jerk-formcodes}
\centering
\small
\begin{tabular}{|l|l|l|}
\hline
\textbf{Categorization} & \textbf{Condition} & \textbf{Applied To} \\
\hline
smooth & jerk $\leq 2.5$ m/s$^3$ & movement\_smoothness \\
shaky & $2.5 <$ jerk $\leq 5.0$ m/s$^3$ & movement\_smoothness \\
unstable & jerk $> 5.0$ m/s$^3$ & movement\_smoothness \\
\hline
\end{tabular}
\end{table}

\subsubsection{Deviation FormCodes}
Measure maximum deviation from expected path or position.

\begin{table}[h]
\caption{Deviation FormCode categorizations.}
\label{tab:deviation-formcodes}
\centering
\small
\begin{tabular}{|l|l|l|l|}
\hline
\textbf{Categorization} & \textbf{Condition} & \textbf{Unit} & \textbf{Applied To} \\
\hline
stable & $0.81 \leq v \leq 1.21$ & ratio & knee\_stability \\
slight\_wobble & $v < 0.81$ & ratio & knee\_stability (valgus) \\
unstable & $v > 1.21$ & ratio & knee\_stability \\
\hline
clean\_arc & $v \leq 0.05$ & m & path\_arc \\
wandering\_arc & $v > 0.05$ & m & path\_arc \\
\hline
no\_swing & $v \leq 0.15$ & normalized & swing\_momentum \\
slight\_swing & $0.15 < v \leq 0.25$ & normalized & swing\_momentum \\
body\_swing & $v > 0.25$ & normalized & swing\_momentum \\
\hline
\end{tabular}
\end{table}

\subsubsection{Symmetry FormCodes}
Measure left-right asymmetry during movement.

\begin{table}[h]
\caption{Symmetry FormCode categorizations.}
\label{tab:symmetry-formcodes}
\centering
\small
\begin{tabular}{|l|l|l|}
\hline
\textbf{Categorization} & \textbf{Condition (m)} & \textbf{Applied To} \\
\hline
level & $v \leq 0.03$ & hip\_shift \\
slight\_shift & $0.03 < v \leq 0.07$ & hip\_shift \\
major\_shift & $v > 0.07$ & hip\_shift \\
\hline
\end{tabular}
\end{table}

\subsubsection{Duration FormCodes}
Measure time spent in specific positions.

\begin{table}[h]
\caption{Duration FormCode categorizations.}
\label{tab:duration-formcodes}
\centering
\small
\begin{tabular}{|l|l|l|}
\hline
\textbf{Categorization} & \textbf{Condition} & \textbf{Applied To} \\
\hline
squeeze & $t \geq 0.5$ s & pause\_at\_top \\
touch\_and\_go & $t < 0.5$ s & pause\_at\_top \\
\hline
\end{tabular}
\end{table}

\section{LLM Prompt Structure}
\label{appendix:llm-prompt}

This appendix documents the data sent to the LLM (Llama 3.3 70B via Cerebras) for generating post-workout summaries.

\subsection{Prompt Components}

The LLM receives two components:

\paragraph{System Prompt}
Defines the AI's role and output requirements:
\begin{verbatim}
You are a friendly and encouraging fitness coach 
summarizing a workout session. You have detailed 
form analysis data including a form score, good reps 
count, and specific issues detected.

Provide a concise summary (3-4 sentences) that:
1. Opens with positive acknowledgment of their effort
2. Mentions total reps and the FORM SCORE
3. If there are issues, mention the most common one 
   as a constructive tip
4. Ends with motivation for their next session
\end{verbatim}

\paragraph{User Prompt}
Contains the workout data in a structured format:
\begin{itemize}
    \item \textbf{Exercise name} and rep counts (total, good)
    \item \textbf{Form score} as a percentage
    \item \textbf{Top issues} with Super FormCode names and descriptions
    \item \textbf{Per-rep feedback sentences} for all repetitions
\end{itemize}

\subsection{Example Request}

The following shows an actual LLM request from a 5-rep bicep curl session:

\begin{verbatim}
{
  "timestamp": "2025-12-12T14:42:58",
  "request_type": "session_summary",
  "user_prompt": "
    Workout: bicep_curls
    Total Reps: 5
    Good Reps: 1
    Form Score: 20%
    
    Top Issues Detected:
    BODY SWING: The user used their back and 
      shoulders to lift the weight.
    
    Per-Rep Feedback:
    Rep 1: You achieved a full muscle contraction 
      at the top of the curl.; Your elbow stayed 
      stable by your side - excellent form!; 
      The lift was explosive...; There was slight 
      body movement during the curl.
    Rep 2: You achieved a full muscle contraction...;
      Significant body swing detected - you're using
      momentum instead of muscle.
    Rep 3: [similar feedback sentences]
    Rep 4: [similar feedback sentences]
    Rep 5: You achieved a full muscle contraction...;
      The lifting speed was well-controlled.;
      Significant body swing detected...
  ",
  "response": "Great job pushing through those bicep 
    curls - I can tell you gave it your all! You 
    completed 5 reps with a form score of 20%, 
    indicating there's room for improvement in 
    your technique. One area to focus on for next 
    time is reducing body swing..."
}
\end{verbatim}

\subsection{Design Rationale}

Including all per-rep FormScript sentences provides the LLM with:
\begin{enumerate}
    \item \textbf{Rep-by-rep progression}: The LLM can observe if form improved or degraded over the session.
    \item \textbf{Richer context}: Enables more personalized summaries that reference specific patterns.
    \item \textbf{Consistency}: The LLM summary aligns with the detailed per-rep feedback the user already saw.
\end{enumerate}

All LLM requests are logged to \texttt{llm\_logs/} for debugging and analysis.
